\chapter{Polimorfismo}

El polimorfismo es la capacidad de ofrecer una interfaz para distintos tipos, de manera que un tipo polimórfico es al que se le pueden aplicar operaciones con distintos tipos. Existen distintos tipos de polimorfismo:


\begin{itemize}
\item \textbf{Polimorfismo ad-hoc\cite{strachey2000fundamental}. } Es cuando una función tiene un conjunto de implementaciones distintas sobre un rango de tipos de datos y sus combinaciones. Este tipo de polimorfismo es soportado en muchos lenguajes por medio de la sobrecarga y es también conocido como \textbf{polimorfismo estático}.

\item \textbf{Polimorfismo de subtipo o de inclusión\footnote{''Polymorphic types are types whose operations are applicable to values of more than one type.''}\cite{cardelli1985understanding}   .} Es el tipo de polimorfismo más común, en el que un conjunto de instancias de distintas clases están relacionadas por una superclase. Tan común que es lo que muchas veces se explica como polimorfismo. También conocido como \textbf{polimorfismo dinámico}.

\item \textbf{Polimorfismo paramétrico\cite{strachey2000fundamental} .} Cuando se escribe código sin especificar el tipo que va a ser usado. En POO es conocido como programación genérica. En programación funcional es llamado simplemente polimorfismo.
\end{itemize}





