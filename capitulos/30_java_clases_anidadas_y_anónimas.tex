\chapter{Java: Clases anidadas y anónimas}
\section{Clases anidadas}

\textbf{Las clases anidadas son clases definidas dentro del alcance de otras clases. }Existen dos tipos de clases anidadas\footnote{Fuente: \href{Fuente: https://docs.oracle.com/javase/tutorial/java/javaOO/nested.html}{Nested class}}:

\begin{itemize}
\item Clase anidadas estáticas
\item Clases interiores, las cuales son clases anidadas no estáticas
\end{itemize}

% TABLA con más de una línea de CÓDIGO
\begin{center}
\begin{tabular}{|m{\textwidth}| }
 \hline 
\rowcolor{gray!50} 
\begin{center}
\textbf{Sintaxis} 
\end{center} 
\\ \hline
    \begin{minted} [linenos=false]{java}
class claseExterna {
    ...
    static class ClaseAnidadaEstática {
        ...
    }
    class claseInterior {
        ...
    }
}
    \end{minted}
\\ \hline
\end{tabular}
\end{center}

Una clase anidada es un miembro de la clase que la define, la clase externa. Clases interiores (clases anidadas no estáticas) tienen acceso a otros miembros de la clase externa, inclusive si estos miembros son privados. Clases anidadas estáticas no tienen acceso a los otros miembros de la clase externa. Las clases anidadas, al ser miembros de una clase pueden ser definidas como públicas, protegidas o privadas (privadas a nivel de paquete).

Una clase anidada puede ser 
\begin{itemize}
\item Es una forma de agrupar clases que son usadas en un sólo lugar
\item Incrementa la encapsulación
\item Puede llevar a un código más fácil de leer y/o mantener.
\end{itemize}

\ejemplo
Clases internas.

%longlist  para potencialmente grandes codigos fuentes
\begin{longlisting}
\begin{minted}
    [frame=lines,
framesep=2mm,
baselinestretch=1.2,
bgcolor=lightgray,
fontsize=\footnotesize]
    {java}
public class EstructuraDeDatos {
    
    // Crea un arreglo 
    private final static int TAMAÑO = 15;
    private int[] arregloDeEnteros = new int[TAMAÑO];
    
    public EstructuraDeDatos() {
        for (int i = 0; i < TAMAÑO; i++) {
        	arregloDeEnteros[i] = i;
        }
    }
    
    public void despliegaPares() {
        
        // despliega valores de índices pares del arreglo
    	IteradorDeEstructuraDeDatos iterador = this.new IteradorPares();
        while (iterador.hasNext()) {
            System.out.print(iterador.next() + " ");
        }
        System.out.println();
    }
    
    //interfaz anidada extiende la interfaz Iterator<Interger>
    interface IteradorDeEstructuraDeDatos extends java.util.Iterator<Integer> { } 

    // Clase anidada implementa interfaz IteradorDeEstructuraDeDatos
    private class IteradorPares implements IteradorDeEstructuraDeDatos {
        
        // Inicializa el iterador para el inicio del arreglo
        private int nextIndice = 0;
        
		@Override
        public boolean hasNext() {
            //Verifica si el elemento actual es el último en el arreglo 
            return (nextIndice <= TAMAÑO - 1);
        }        
        
		@Override
        public Integer next() {
            // Registra un valor de un índice par del arreglo
            Integer reValor = Integer.valueOf(arregloDeEnteros[nextIndice]);
            
            // obtiene el siguiente elemento par
            nextIndice += 2;
            return reValor;
        }

		@Override
		public void remove() {
			// implementación es opcional
			throw new UnsupportedOperationException();
			
		}
    }
    
    public static void main(String s[]) {
    	EstructuraDeDatos edd = new EstructuraDeDatos();
        edd.despliegaPares();
    }
}
\end{minted}
\caption{Ejemplo de clases internas.}
% \caption[Long Code Example]{A long code example which will break across pages.}
\label{listing:1}
\end{longlisting}

Variables con el mismo nombre. \ejemplo

%longlist  para potencialmente grandes codigos fuentes
\begin{longlisting}
\begin{minted}
    [frame=lines,
framesep=2mm,
baselinestretch=1.2,
bgcolor=lightgray,
fontsize=\footnotesize]
    {java}
public class PruebaOcultarVariables {

    public int x = 0;

    class PrimerNivel {

        public int x = 1;

        void métodoEnPrimerNivel(int x) {
            System.out.println("x = " + x);
            System.out.println("this.x = " + this.x);
            System.out.println("PruebaOcultarVariables.this.x = " 
            + PruebaOcultarVariables.this.x); //Despliega x de la clase externa (x=0)
        }
    }

    public static void main(String... args) {
    	PruebaOcultarVariables pov = new PruebaOcultarVariables();
    	PruebaOcultarVariables.PrimerNivel pn = pov.new PrimerNivel();
    	pn.métodoEnPrimerNivel(25);
    }
}
\end{minted}
\caption{Ejemplo variables con el mismo nombre en clases internas .}
% \caption[Long Code Example]{A long code example which will break across pages.}
\label{listing:1}
\end{longlisting}

\section{Clases anónimas}

\textbf{Una clase anónima es una clase anidada que no tiene un nombre.} Combina  en un solo paso la definición de la clase anidada y la instanciación de la misma.

Una clase anónima se ocupa cuando una clase solo se requiere una vez\footnote{\href{https://docs.oracle.com/javase/tutorial/java/javaOO/anonymousclasses.html}{Anonymouse classes}}.

Las clases anónimas son compiladas con el nombre de las clase que las contienen seguida de \$ y un número consecutivo. Por ejemplo: \textit{Clasecontenedora\$1.class}

Un código de clases anónimas. \ejemplo

%longlist  para potencialmente grandes codigos fuentes
\begin{longlisting}
\begin{minted}
    [frame=lines,
framesep=2mm,
baselinestretch=1.2,
bgcolor=lightgray,
fontsize=\footnotesize]
    {java}
public class ClasesAnónimasHolaMundo {
	  
    interface HolaMundo {
        public void saludar();
        public void saludarAlguien(String alguien);
    }
  
    public void decirHola() {
        
        class SaludarEnInglés implements HolaMundo {
            String nombre = "world";
            public void saludar() {
            	saludarAlguien("world");
            }
            public void saludarAlguien(String alguien) {
                nombre = alguien;
                System.out.println("Hello " + nombre);
            }
        }
      
        HolaMundo saludarEnInglés = new SaludarEnInglés();
        
        HolaMundo saludarEnFrances = new HolaMundo() {
            String nombre = "tout le monde";
            public void saludar() {
            	saludarAlguien("tout le monde");
            }
            public void saludarAlguien(String alguien) {
                nombre = alguien;
                System.out.println("Salut " + nombre);
            }
        };
        
        HolaMundo saludarEnEspañol = new HolaMundo() {
            String nombre = "mundo";
            public void saludar() {
            	saludarAlguien("mundo");
            }
            public void saludarAlguien(String alguien) {
                nombre = alguien;
                System.out.println("Hola, " + nombre);
            }
        };

        saludarEnInglés.saludar();
        saludarEnFrances.saludarAlguien("Juliette");
        saludarEnEspañol.saludar();
    }

    public static void main(String... args) {
    		ClasesAnónimasHolaMundo miApl =
            new ClasesAnónimasHolaMundo();
        	miApl.decirHola();
    }            
}
\end{minted}
\caption{Ejemplo de clases anónimas.}
% \caption[Long Code Example]{A long code example which will break across pages.}
\label{listing:1}
\end{longlisting}


\fi
%%%%%%%%%%
% JAVA_END
%%%%%%%%%%


%%%%%%%%%%
% JAVA
%%%%%%%%%%
\ifjava

