\chapter{Amistad en C++}

En C++ existe el concepto de \textbf{amistad}.  Aunque puede ser considerado por algunos como una intrusión a la encapsulación o a la privacidad de los datos: 

% TABLA con más de una línea 
\begin{center}
\begin{tabular}{|m{\textwidth}| }
 \hline 
\rowcolor{gray!50} 
\begin{center}
\textbf{Concepto} 
\end{center} 
\\ \hline
\textit{"... la amistad corrompe el ocultamiento de información y debilita el valor del enfoque de diseño orientado a objetos"}\cite{deitel2006c++}    
\\ \hline
\end{tabular}
\end{center}

Un amigo de una clase es una función u otra clase que no es miembro de la clase, pero que tiene permiso de usar los miembros públicos y privados de la clase\footnote{También tiene acceso a los miembros protegidos que se verán más adelante.}. 

Es importante señalar que el ámbito de una función amiga no es el de la clase, y por lo tanto los amigos no son llamados con los operadores de acceso de miembros.

% TABLA con más de una línea de CÓDIGO
\begin{center}
\begin{tabular}{|m{\textwidth}| }
 \hline 
\rowcolor{gray!50} 
\begin{center}
\textbf{Sintaxis} 
\end{center} 
\\ \hline
Sintaxis para una función amiga:
    \begin{minted} [linenos=false]{cpp}
class <nombreClase> {
	friend <tipo> <metodo>();
...
public:
	...
};
    \end{minted}
\\
Sintaxis para una clase amiga:
  \begin{minted} [linenos=false]{cpp}
class <nombreClase> {
	friend <nombreClaseAmiga>;
...
public:
	...
};
    \end{minted}
\\ \hline
\end{tabular}
\end{center}

	Las funciones o clases amigas no son privadas ni públicas (o protegidas), pueden ser colocadas en cualquier parte de la definición de la clase, pero se acostumbra que sea al principio.[Deitel, 1995]

	Como la amistad entre personas, esta es \textbf{concedida} y no tomada. Si la clase B quiere ser amigo de la clase A, la clase A debe declarar que la clase B es su amiga.

	La amistad \textbf{no es simétrica ni transitiva}: si la clase A es un amigo de la clase B, y la clase B es un amigo de la clase C, no implica:

\begin{itemize}
\item     Que la clase B sea un amigo de la clase A.
\item     Que la clase C sea un amigo de la clase B.
\item     Que la clase A sea un amigo de la clase C.
\end{itemize}

El concepto de amistad no esta implementado en otros lenguajes aunque el nivel protegido  permite un cierto nivel de acceso miembros de clases del mismo módulo an algunos lenguajes.

\textcolor{blue}{Ejemplo:}

%longlist  para potencialmente grandes codigos fuentes
\begin{longlisting}
\begin{minted}
    [frame=lines,
framesep=2mm,
baselinestretch=1.2,
bgcolor=lightgray,
fontsize=\footnotesize]
    {cpp}
//Ejemplo de funcion amiga con acceso a miembros privados
#include <iostream>

using namespace std;

class ClaseX{
	friend void setX(ClaseX &, int);  //declaración friend
	public:
	ClaseX(){
		x=0;
	}
	void print() const {
		cout<<x<<endl;
	}
	private:
	int x;
};

void setX(ClaseX &c, int val){
	c.x=val;	//es legal el acceso a miebros privados por amistad.
}

int main(){
	ClaseX pr;

	cout<<"pr.x después de instanciación : ";
	pr.print();
	cout<<"pr.x después de la llamada a la función amiga setX : ";
	setX(pr, 10);
	pr.print();
}
\end{minted}
\caption{Ejemplo de funciones amigas en C++.}
% \caption[Long Code Example]{A long code example which will break across pages.}
\label{listing:1}
\end{longlisting}

%longlist  para potencialmente grandes codigos fuentes
\begin{longlisting}
\begin{minted}
    [frame=lines,
framesep=2mm,
baselinestretch=1.2,
bgcolor=lightgray,
fontsize=\footnotesize]
    {cpp}
//ejemplo 2 de funciones amigas
#include <iostream>
using namespace std;

class Linea;

class Recuadro {
	friend int mismoColor(Linea, Recuadro);
	
	private:
	int color; //color del recuadro
	int xsup, ysup; //esquina superior izquierda
	int xinf, yinf; //esquina inferior derecha
	
	public:
	void ponColor(int);
	void definirRecuadro(int, int, int, int);
};

class Linea{
	friend int mismoColor(Linea, Recuadro);
	
	private:
	int color;
	int xInicial, yInicial;
	int lon;
	
	public:
	void ponColor(int);
	void definirLinea(int, int, int);
};

int mismoColor(Linea l, Recuadro r){
	if(l.color==r.color)
		return 1;
	return 0;
}

//métodos de la clase Recuadro
void Recuadro::ponColor(int c) {
	color=c;
}

void Recuadro::definirRecuadro(int x1, int y1, int x2, int y2) {
	xsup=x1;
	ysup=y1;
	xinf=x2;
	yinf=y2;
}

//métodos de la clase Linea
void Linea::ponColor(int c) {
	color=c;
}

void Linea::definirLinea(int x, int y, int l) {
	xInicial=x;
	yInicial=y;
	lon=l;
}

int main(){
	Recuadro r;
	Linea l;

	r.definirRecuadro(10, 10, 15, 15);
	r.ponColor(3);
	l.definirLinea(2, 2, 10);
	l.ponColor(4);
	if(!mismoColor(l, r))
		cout<<"No tienen el mismo color"<<endl;
	//se ponen en el mismo color
	l.ponColor(3);
	if(mismoColor(l, r))
		cout<<"Tienen el mismo color";
	return 0;
}
\end{minted}
\caption{Ejemplo 2 de funciones amigas en C++.}
% \caption[Long Code Example]{A long code example which will break across pages.}
\label{listing:1}
\end{longlisting}






