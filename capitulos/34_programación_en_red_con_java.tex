\chapter{Programación en Red con Java}

Java, como un lenguaje de programación moderno, provee de clases para el manejo de información en red. De hecho, el uso de otras tecnologías como JDBC involucra el acceso a bases de datos locales y remotas de forma prácticamente transparente.

\section{Paquete \textit{java.net}}

Dentro del paquete \textit{java.net} se tienen un conjunto de clases que dan soporte a los diversos protocolos de comunicación de Internet, conocido como paquete de protocolos de Internet, dentro de los cuales se encuentran:

 
\begin{enumerate}
\item IP. \textit{Internet Protocol}.
\item TCP. \textit{Transmission Control Protocol}.
\item UDP. \textit{User Datagram Protocol}.
\end{enumerate}

La mayor parte de las aplicaciones están basadas en los protocolos TCP/IP, las cuales muchas veces hacen uso de otros protocolos intermediarios entre TCP/IP y la aplicación.

La tabla \ref{tab:protocolos} muestra una lista de protocolos de uso común en Internet.

% Please add the following required packages to your document preamble:
% \usepackage{graphicx}
% \usepackage[table,xcdraw]{xcolor}
% If you use beamer only pass "xcolor=table" option, i.e. \documentclass[xcolor=table]{beamer}
\begin{table}[]
\caption{Protocolos comunes}
\label{tab:protocolos}
\resizebox{\textwidth}{!}{%
\begin{tabular}{|l|l|l|}
\hline
\rowcolor[HTML]{C0C0C0} 
\multicolumn{1}{|c|}{\cellcolor[HTML]{C0C0C0}\textbf{Acrónimo}} & \multicolumn{1}{c|}{\cellcolor[HTML]{C0C0C0}\textbf{Nombre}} & \multicolumn{1}{c|}{\cellcolor[HTML]{C0C0C0}\textbf{Descripción}} \\ \hline
HTTP                                                            & \textit{HyperText Transport Protocol}                        & Protocolo de hipertexto. Es la base del World Wide Web.           \\ \hline
FTP                                                             & \textit{File Transfer,Protocol}                              & Protocolo de transferencia de archivos.                           \\ \hline
POP3                                                            & \textit{Post Office Protocol}                                & Protocolo que permite el acceso al correo electrónico.            \\ \hline
SMTP                                                            & \textit{Simple Mail Transfer Protocol}                       & Protocolo para transferencia de correo electrónico.               \\ \hline
NNTP                                                            & \textit{Network News Transfer Protocol}                      & Protocolo para grupos de noticias (news)                          \\ \hline
\end{tabular}%
}
\end{table}

El paquete \textit{java.net} cuenta hasta la versión 1.4 del jdk con 6 interfaces, 27 clases y 11 excepciones. Las clases más usadas son:

 
\begin{enumerate}
\item \textit{URL}. Representa un URL de Internet.
\item \textit{URLConnection}. Es un complemento de URL (no una subclase de ella). Cubre algunas operaciones más complejas.
\item \textit{Socket}. Establece conexiones TCP/IP.
\item \textit{DatagramPacket}. Establece conexiones de tipo UDP.
\item \textit{InetAddress}. Representa una dirección IP.
\end{enumerate}

\subsection{Clase \textit{URL}}

Esta clase permite crear instancias que almacenen direcciones de recursos en Internet\footnote{URL. \textit{Uniform Resource Locator}}. Este recurso puede ser un archivo, directorio, o inclusive una consulta a un motor de búsqueda. En caso de que el URL tenga una sintaxis incorrecta se lanza una excepción \textit{MalformedURLException}.

\ejemplo  %ACTUALIZAR

%longlist  para potencialmente grandes codigos fuentes
\begin{longlisting}
\begin{minted}
    [frame=lines,
framesep=2mm,
baselinestretch=1.2,
bgcolor=lightgray,
fontsize=\footnotesize]
    {java}
//Ejemplo de uso de URL
import java.applet.*;
import java.net.*;
import java.awt.*;

public class URLEjemplo extends Applet {

   URL utm = null;
 
   public void init()
     {

       try {
           utm = new URL("http://www.utm.mx");
         } 
       catch (MalformedURLException e)
         {
           System.out.println("Error:" + e.getMessage());
         }
     }

   public boolean mouseDown(Event evt, int x, int y)
     {
       getAppletContext().showDocument(utm);
       return(true);
     }
  }
\end{minted}
\caption{Ejemplo de clase URL.}
% \caption[Long Code Example]{A long code example which will break across pages.}
\label{listing:1}
\end{longlisting}

Este programa detecta un \textit{click} del ratón sobre el área del \textit{applet} y como acción asociada despliega una página de html en el navegador\footnote{No se pruebe en el \textit{appletviewer} ya que este no despliega más que el \textit{applet} y no muestra la página html.}.

Este es sólo un ejemplo, pero la clase URL es usada por todos aquellos programas que requieran del uso de direcciones de recursos de Internet.

\subsection{Clase \textit{InetAddress}}

Como se mencionó antes, esta clase representar una dirección IP. La clase no maneja atributos ni constructores. Ofrece en cambio métodos de acceso para operaciones comunes de Internet.

\ejemplo
%longlist  para potencialmente grandes codigos fuentes
\begin{longlisting}
\begin{minted}
    [frame=lines,
framesep=2mm,
baselinestretch=1.2,
bgcolor=lightgray,
fontsize=\footnotesize]
    {java}
//obtiene la dirección IP de la máquina local

import java.net.*;

public class ObtenIPLocal {

    public static void main(String args[]) {
        InetAddress IPLocal=null;

        try {
            IPLocal= InetAddress.getLocalHost();
        }catch (UnknownHostException e) {}

        System.out.println(IPLocal);
    }
}
\end{minted}
\caption{Ejemplo de \textit{InetAddress}, obtiene la dirección IP de la máquina local.}
% \caption[Long Code Example]{A long code example which will break across pages.}
\label{listing:1}
\end{longlisting}

El programa anterior usa una instancia de \textit{InetAddress} para obtener la dirección de la máquina local mediante el método $getLocalHost()$ de la clase. Es un ejemplo muy simple del uso de la clase.

El siguiente programa  recibe como parámetro el nombre de un servidor y obtiene la dirección asociada a ese nombre.

\ejemplo

%longlist  para potencialmente grandes codigos fuentes
\begin{longlisting}
\begin{minted}
    [frame=lines,
framesep=2mm,
baselinestretch=1.2,
bgcolor=lightgray,
fontsize=\footnotesize]
    {java}
//identifica la direccion IP asociada al host
import java.net.InetAddress;
import java.net.UnknownHostException;
import java.lang.System;

public class NSLookupApp {
 public static void main(String args[]) {
  try {
   if(args.length!=1){
    System.out.println("Sintaxis: java NSLookupApp nombreServidor");
    return;
   }
   InetAddress host = InetAddress.getByName(args[0]);
   String hostName = host.getHostName();
   System.out.println("Nombre servidor: "+hostName);
   System.out.println("Direccion IP: "+host.getHostAddress());
  }catch(UnknownHostException ex) {
   System.out.println("Servidor desconocido");
   return;
  }
 }
}
\end{minted}
\caption{Ejemplo de \textit{InetAddress}, identifica la dirección IP asociada al \textit{host}.}
% \caption[Long Code Example]{A long code example which will break across pages.}
\label{listing:1}
\end{longlisting}
%\\
Ejecutando por ejemplo:

\mint[linenos=false]{bash}| \$java NSLookupApp www.utm.mx |


\subsection{Clase \textit{Socket}}

Esta clase se usa para la implementación de \textit{sockets} de cliente basados en una conexión.  La aplicación cliente debe comúnmente iniciar la conexión de \textit{sockets} hacia el servidor.

Una instancia de la clase \textit{Socket} es creada con el constructor recibiendo como parámetros por lo general el número IP o nombre de dominio del servidor y el puerto del servidor, creando una conexión a un puerto y \textit{host} de destino.

Un \textit{socket} se puede crear de la siguiente forma:

\mint[linenos=false]{java}| miSocket = new Socket ("mixteco.utm.mx", 1111); |

Esta línea de código tiene que estar dentro de un segmento \textit{try} para recibir una excepción en caso de que se produzca. 

Algunos métodos importantes de la clase \textit{Socket}:

\begin{itemize}
\item \textit{getInetAddress()}.  Obtiene la dirección IP del servidor destino.
\item \textit{getPort()}. Obtiene el puerto del servidor destino.
\item \textit{getLocalAddress()}. Obtiene la dirección IP local.
\item \textit{getLocalPort()}. Obtiene el número de puerto local.
\item \textit{getInputStream()}. Para acceder a los flujos de entrada.
\item \textit{getOutputStream()}. Para acceder a los flujos de salida.
\item \textit{close()}. Cerrar el socket cliente.
\end{itemize}

\subsection{Clase ServerSocket}

Esta clase implementa un \textit{socket} del servidor TCP. Una instancia de la clase recibe comúnmente el número de puerto por el cual va a \textbf{escuchar} las solicitudes de conexión del cliente.

Dentro de código para manejo de excepciones se declara un \textit{socket} servidor de la siguiente forma:

\mint[linenos=false]{java}| miServidor = new ServerSocket(1111); |

Algunos métodos importantes de la clase \textit{ServerSocket}:


\begin{itemize}
\item \textit{accept()}. Hace que el \textit{socket} servidor escuche y espere hasta que se establezca una conexión entrante.
\item \textit{getSoTimeout()}. Devuelve el tiempo que va a estar bloqueado el \textit{socket} con respecto a una llamada al método \textit{accept()}.
\item \textit{setSoTimeout()}. Modifica el tiempo de bloqueo del \textit{socket}.
\item \textit{close()}. Cierra el \textit{socket} servidor.
\end{itemize}

Veamos ahora un ejemplo con una clase cliente y otra clase servidor. Este programa aprovecha las características de multihilos de Java creando un hilo cliente y otro servidor.

\ejemplo %ACTUALIZAR

%longlist  para potencialmente grandes codigos fuentes
\begin{longlisting}
\begin{minted}
    [frame=lines,
framesep=2mm,
baselinestretch=1.2,
bgcolor=lightgray,
fontsize=\footnotesize]
    {java}
//Programa cliente servidor con sockets
import java.awt.*;
import java.net.*;
import java.io.*;

class hiloCliente extends Thread {

   DataInputStream dis = null;
   Socket s = null;

   public hiloCliente() {
        try {
           //se crea socket con dirección de máquina local
           s = new Socket("127.0.0.1", 2525);
           dis = new DataInputStream(s.getInputStream());
        } 
        catch (IOException e)
        {
            System.out.println("Error: " + e);
        }
   }

   public void run()
   {
       while (true)
       {
            try {
               String mensaje = dis.readLine();
               if (mensaje == null)
                 break;
               System.out.println(mensaje);
           } 
           catch (IOException e)
           {
               System.out.println("Error: " + e);
           }
       }
   }
}

public class clienteYServidor extends Frame {
   static ServerSocket servidor = null;

   public boolean handleEvent (Event evt)
   {
       if (evt.id == Event.WINDOW_DESTROY)
       {
           System.exit(0);
       }
       return super.handleEvent (evt);
   }

   public boolean mouseDown(Event evt, int x, int y)
   {
       new hiloCliente().start(); // Iniciar el hilo cliente
       return(true);
   }

   public static void main(String args[])
   {
      clienteYServidor f = new clienteYServidor();
      f.resize (200, 200);
      f.show();
      try {
            //genera el socket servidor
          servidor = new ServerSocket(2525);
      } 
      catch (IOException e)
      {
          System.out.println("Error: " + e);
      }
      while (true)
      {
          Socket s = null;
          try {
              s = servidor.accept();
          } 
          catch (IOException e)
          {
              System.out.println("Error: " + e);
          }

          try {
            PrintStream ps = new  PrintStream(s.getOutputStream());
            ps.println("Hola, Mundo");
            ps.flush();
            s.close();
          } 
          catch (IOException e)
          {
             System.out.println("Error: " + e);
          }
      }
  }
}
\end{minted}
\caption{Ejemplo de programa cliente servidor con sockets .}
% \caption[Long Code Example]{A long code example which will break across pages.}
\label{listing:1}
\end{longlisting}

El programa muestra a un hilo cliente solicitando una cadena de un flujo de datos al servidor: cada vez que se da \textit{click} sobre la ventana el servidor inicia a un hilo cliente que a su vez se comunica con el servidor.

En el \textbf{sección complementaria uno }se muestra otro ejemplo de uso de la clase \textit{Socket} para crear un cliente y en la \textbf{sección complementaria dos} se muestra el código correspondiente al servidor. Estos programas se comunican entre sí: el cliente es capaz de enviar una cadena y recibirla de vuelta modificada por el servidor. Cada uno corre de manera independiente e idealmente debería ser probado en distintas máquinas en red.

Ejemplo de ejecución del \textbf{cliente}:

\begin{minted}{bash}
$ java PortTalkApp yodocono.utm.mx 1234
Conectando a: yodocono.utm.mx puerto 1234.
servidor destino yodocono.utm.mx.
IP servidor destino 192.100.170.5.
numero de puerto servidor destino 1234.
servidor Local iec23.
IP servidor Local 192.100.170.33.
numero de puerto servidor Local 2162.
Enviar, recibir, o salir (E/R/S): e
Hola yodocono
Enviar, recibir, o salir (E/R/S): r
***onocodoy aloH
\end{minted}

Ejemplo de ejecución del \textbf{servidor}:

\begin{minted}{bash}
$ java ReverServerApp
Servidor escuchando en puerto: 1234.
Aceptando conexion a iec23.utm.mx en puerto 2162.
Recibido: Hola yodocono
Enviado: onocodoy aloH
\end{minted}


\subsection{Clase \textit{DatagramSocket}}

Esta clase es el punto de entrada de todas las acciones sobre datagramas UDP\footnote{UDP es un protocolo que carece de conexión y que permite que los programas de aplicación intercambien información por medio de trozos de información a los que se conoce datagramas.}. Sería el equivalente a la clase \textit{Socket} y \textit{ServerSocket} para el protocolo TCP, ya que implementa los \textit{sockets} cliente y servidor.

	Principales métodos\footnote{Algunos métodos importantes no se mencionan porque son comunes a los proporcionados por otras clases con anterioridad.} de la clase \textit{DatagramSocket}:


\begin{itemize}
\item \textit{send()}. Enviar un datagrama a través del \textit{socket}.
\item \textit{receive()}. Recibir un datagrama a través del \textit{socket}.
\item \textit{close()}. Cerrar el \textit{socket}.
\end{itemize}

\subsection{Clase \textit{DatagramPacket}}

Esta clase representa un paquete de datos recibido o enviado mediante un \textit{socket} a través del protocolo UDP. Se le considera una clase de bajo nivel que sólo resulta útil para aplicaciones que deben leer o escribir datos según un formato específico, pero que no necesitan garantizar la integridad de la llamada.

Cada datagrama es enviado de una máquina a otra a partir de la información del paquete. Si un conjunto de paquetes son enviados hacia una máquina estos podrían ser enviados por caminos distintos y tener un orden de llegada distinto.

A continuación se muestra la salida generada por dos programas que se detallan en las \textbf{secciones complementarias tres y cuatro}. El programa \textit{TimeServerApp} esta a la espera de solicitudes de \textbf{tiempo} o \textbf{salida} y de acuerdo a estas solicitudes enviará al cliente la fecha y hora o provocará la finalización del servidor. 

Ejemplo de ejecución de \textit{TimeServerApp}:

\begin{minted}{bash}
$ java TimeServerApp
iec23: TimeServer escuchando el puerto 2345.

Recibiendo un datagrama desde iec23.utm.mx puerto 1188.
Contenido del datagrama: tiempo
Enviando: Fri Jun 02 17:20:58 GMT-05:00 2000 a iec23.utm.mx al puerto 1188.

Recibiendo un datagrama desde iec23.utm.mx puerto 1188.
Contenido del datagrama: tiempo
Enviando: Fri Jun 02 17:20:59 GMT-05:00 2000 a iec23.utm.mx al puerto 1188.

Recibiendo un datagrama desde iec23.utm.mx puerto 1188.
Contenido del datagrama: tiempo
Enviando: Fri Jun 02 17:20:59 GMT-05:00 2000 a iec23.utm.mx al puerto 1188.

Recibiendo un datagrama desde iec23.utm.mx puerto 1188.
Contenido del datagrama: tiempo
Enviando: Fri Jun 02 17:20:59 GMT-05:00 2000 a iec23.utm.mx al puerto 1188.

Recibiendo un datagrama desde iec23.utm.mx puerto 1188.
Contenido del datagrama: tiempo
Enviando: Fri Jun 02 17:20:59 GMT-05:00 2000 a iec23.utm.mx al puerto 1188.

Recibiendo un datagrama desde iec23.utm.mx puerto 1188.
Contenido del datagrama: salida
Enviando: Fri Jun 02 17:21:00 GMT-05:00 2000 a iec23.utm.mx al puerto 1188.
\end{minted}

Por su parte, el programa \textit{GetTimeApp} envía cinco solicitudes de "tiempo"  y una de "salida" al servidor.

Ejemplo de ejecución de \textit{GetTimeApp}:

\begin{minted}{java}
$ java GetTimeApp

Envia peticion de tiempo a iec23 al puerto 2345.
Recibiendo un datagrama desde iec23.utm.mx at port 2345.
El datagrama contiene los sig. datos: Fri Jun 02 17:20:58 GMT-05:00 2000

Envia peticion de tiempo a iec23 al puerto 2345.
Recibiendo un datagrama desde iec23.utm.mx at port 2345.
El datagrama contiene los sig. datos: Fri Jun 02 17:20:59 GMT-05:00 2000

Envia peticion de tiempo a iec23 al puerto 2345.
Recibiendo un datagrama desde iec23.utm.mx at port 2345.
El datagrama contiene los sig. datos: Fri Jun 02 17:20:59 GMT-05:00 2000

Envia peticion de tiempo a iec23 al puerto 2345.
Recibiendo un datagrama desde iec23.utm.mx at port 2345.
El datagrama contiene los sig. datos: Fri Jun 02 17:20:59 GMT-05:00 2000

Envia peticion de tiempo a iec23 al puerto 2345.
Recibiendo un datagrama desde iec23.utm.mx at port 2345.
El datagrama contiene los sig. datos: Fri Jun 02 17:20:59 GMT-05:00 2000
\end{minted}

En este ejemplo se asume el funcionamiento en una sola máquina pues se toma la dirección de la máquina local, pero modificarlo para probarlo en máquinas distintas no debe ser problema.

\section{Complemento 1. \textit{PortTalkApp}}

%longlist  para potencialmente grandes codigos fuentes
\begin{longlisting}
\begin{minted}
    [frame=lines,
framesep=2mm,
baselinestretch=1.2,
bgcolor=lightgray,
fontsize=\footnotesize]
    {java}
//Ejemplo de uso de la clase Socket
import java.lang.System;
import java.net.Socket;
import java.net.InetAddress;
import java.net.UnknownHostException;
import java.io.*;

public class PortTalkApp {
 public static void main(String args[]){
  PortTalk portTalk = new PortTalk(args);
  portTalk.displayDestinationParameters();
  portTalk.displayLocalParameters();
  portTalk.chat();
  portTalk.shutdown();
 }
}

class PortTalk {
 Socket connection;
 DataOutputStream outStream;
 BufferedReader inStream;
 public PortTalk(String args[]){
  if(args.length!=2) error("Sintaxis: java PortTalkApp <servidor> <puerto>");
  String destination = args[0];
  int port = 0;
  try {
   port = Integer.valueOf(args[1]).intValue();
  }catch (NumberFormatException ex){
   error("N£mero de puerto inv lido");
  }
  try{
   connection = new Socket(destination,port);
  }catch (UnknownHostException ex){
   error("Servidor desconocido");
  }
  catch (IOException ex){
   error("Error E/S: al crear el socket");
  }
  try{
   inStream = new BufferedReader(
    new InputStreamReader(connection.getInputStream()));
   outStream = new DataOutputStream(connection.getOutputStream());
  }catch (IOException ex){
   error("Error E/S: obteniendo el flujo");
  }
  System.out.println("Conectando a: "+destination+" puerto "+port+".");
 }
 public void displayDestinationParameters(){
  InetAddress destAddress = connection.getInetAddress();
  String name = destAddress.getHostName();
  byte ipAddress[] = destAddress.getAddress();
  int port = connection.getPort();
  displayParameters("servidor destino ",name,ipAddress,port);
 }
 public void displayLocalParameters(){
  InetAddress localAddress = null;
  try{
   localAddress = InetAddress.getLocalHost();
  }catch (UnknownHostException ex){
   error("Error obteniendo informaci¢n de servidor local");
  }
  String name = localAddress.getHostName();
  byte ipAddress[] = localAddress.getAddress();
  int port = connection.getLocalPort();
  displayParameters("servidor Local ",name,ipAddress,port);
 }
 public void displayParameters(String s,String name,byte ipAddress[],int port){
  System.out.println(s+name+".");
  System.out.print("IP "+s);
  for(int i=0;i<ipAddress.length;++i)
   System.out.print((ipAddress[i]+256)%256+".");
  System.out.println();
  System.out.println("numero de puerto "+s+port+".");
 }
 public void chat(){
  BufferedReader keyboardInput = new BufferedReader(
   new InputStreamReader(System.in));
  boolean finished = false;
  do {
   try{
    System.out.print("Enviar, recibir, o salir (E/R/S): ");
    System.out.flush();
    String line = keyboardInput.readLine();
    if(line.length()>0){
     line=line.toUpperCase();
     switch (line.charAt(0)){
     case 'E':
      String sendLine = keyboardInput.readLine();
      outStream.writeBytes(sendLine);
      outStream.write(13);
      outStream.write(10);
      outStream.flush();
      break;
     case 'R':
      int inByte;
      System.out.print("***");
      while ((inByte = inStream.read()) != '\n')
      System.out.write(inByte);
      System.out.println();
      break;
     case 'S':
      finished=true;
      break;
     default:
      break;
     }
    }
   }catch (IOException ex){
    error("Error leyendo del teclado o socket");
   }
  } while(!finished);
 }
 public void shutdown(){
  try{
   connection.close();
  }catch (IOException ex){
   error("Error e/S cerrando socket");
  }
 }
 public void error(String s){
  System.out.println(s);
  System.exit(1);
 }
}
\end{minted}
\caption{Ejemplo de uso de la clase Socket. \textit{PortTalkApp}.}
% \caption[Long Code Example]{A long code example which will break across pages.}
\label{listing:1}
\end{longlisting}

\section{Complemento 2. \textit{ServerSocket}}

%longlist  para potencialmente grandes codigos fuentes
\begin{longlisting}
\begin{minted}
    [frame=lines,
framesep=2mm,
baselinestretch=1.2,
bgcolor=lightgray,
fontsize=\footnotesize]
    {java}
//ejemplo de uso de la clase ServerSocket
import java.lang.System;
import java.net.ServerSocket;
import java.net.Socket;
import java.io.*;

public class ReverServerApp {
 public static void main(String args[]){
  try{
   ServerSocket server = new ServerSocket(1234);
   int localPort = server.getLocalPort();
   System.out.println("Servidor escuchando en puerto: "+localPort+".");
   Socket client = server.accept();
   String destName = client.getInetAddress().getHostName();
   int destPort = client.getPort();
   System.out.println("Aceptando conexion a "+destName+" en puerto "+
    destPort+".");
   BufferedReader inStream = new BufferedReader(
    new InputStreamReader(client.getInputStream()));
   DataOutputStream outStream = new DataOutputStream(client.getOutputStream());
   boolean finished = false;
   do {
    String inLine = inStream.readLine();
    System.out.println("Recibido: "+inLine);
    if(inLine.equalsIgnoreCase("salir")) finished=true;
    String outLine=new ReverseString(inLine.trim()).getString();
    for(int i=0;i<outLine.length();++i)
     outStream.write((byte)outLine.charAt(i));
    outStream.write(13);
    outStream.write(10);
    outStream.flush();
    System.out.println("Enviado: "+outLine);
   } while(!finished);
   inStream.close();
   outStream.close();
   client.close();
   server.close();
  }catch (IOException ex){
   System.out.println("excepcion: IOException .");
  }
 }
}
class ReverseString {
 String s;
 public ReverseString(String in){
  int len = in.length();
  char outChars[] = new char[len];
  for(int i=0;i<len;++i)
   outChars[len-1-i]=in.charAt(i);
  s = String.valueOf(outChars);
 }
 public String getString(){
  return s;
 }
}
\end{minted}
\caption{Ejemplo de uso de la clase \textit{ServerSocket}.}
% \caption[Long Code Example]{A long code example which will break across pages.}
\label{listing:1}
\end{longlisting}


\section{Complemento 3. \textit{TimeServerApp}}

%longlist  para potencialmente grandes codigos fuentes
\begin{longlisting}
\begin{minted}
    [frame=lines,
framesep=2mm,
baselinestretch=1.2,
bgcolor=lightgray,
fontsize=\footnotesize]
    {java}
//ejemplo de escucha de un socket UDP
import java.lang.System;
import java.net.DatagramSocket;
import java.net.DatagramPacket;
import java.net.InetAddress;
import java.io.IOException;
import java.util.Date;

public class TimeServerApp {
 public static void main(String args[]){
  try{
   DatagramSocket socket = new DatagramSocket(2345);
   String localAddress = InetAddress.getLocalHost().getHostName().trim();
   int localPort = socket.getLocalPort();
   System.out.print(localAddress+": ");
   System.out.println("TimeServer escuchando el puerto "+localPort+".");
   int bufferLength = 256;
   byte outBuffer[];
   byte inBuffer[] = new byte[bufferLength];
   DatagramPacket outDatagram;
   DatagramPacket inDatagram = new DatagramPacket(inBuffer,inBuffer.length);
   boolean finished = false;
   do {
    socket.receive(inDatagram);
    InetAddress destAddress = inDatagram.getAddress();
    String destHost = destAddress.getHostName().trim();
    int destPort = inDatagram.getPort();
    System.out.println("\nRecibiendo un datagrama desde "+destHost+" puerto "+
     destPort+".");
    String data = new String(inDatagram.getData()).trim();
    System.out.println("Contenido del datagrama: "+data);
    if(data.equalsIgnoreCase("salida")) finished=true;
    String time = new Date().toString();
    outBuffer=time.getBytes();
    outDatagram = new DatagramPacket(outBuffer,outBuffer.length,destAddress,
     destPort);
    socket.send(outDatagram);
    System.out.println("Enviando: "+time+" a "+destHost+" al puerto "+destPort+".");
   } while(!finished);
  }catch (IOException ex){
   System.out.println("Excepcion: IOException");
  }
 }
}

\end{minted}
\caption{Ejemplo de escucha de un socket UDP. \textit{TimeServerApp}}
% \caption[Long Code Example]{A long code example which will break across pages.}
\label{listing:1}
\end{longlisting}

\section{Complemento 4. \textit{GetTimeApp}}

%longlist  para potencialmente grandes codigos fuentes
\begin{longlisting}
\begin{minted}
    [frame=lines,
framesep=2mm,
baselinestretch=1.2,
bgcolor=lightgray,
fontsize=\footnotesize]
    {java}
//ejemplo de uso de datagramas
import java.lang.System;
import java.net.DatagramSocket;
import java.net.DatagramPacket;
import java.net.InetAddress;
import java.io.IOException;

public class GetTimeApp {
 public static void main(String args[]){
  try{
   DatagramSocket socket = new DatagramSocket();
   InetAddress localAddress = InetAddress.getLocalHost();
   String localHost = localAddress.getHostName();
   int bufferLength = 256;
   byte outBuffer[];
   byte inBuffer[] = new byte[bufferLength];
   DatagramPacket outDatagram;
   DatagramPacket inDatagram = new DatagramPacket(inBuffer,inBuffer.length);
   for(int i=0;i<5;++i){
    outBuffer = new byte[bufferLength];
    outBuffer = "tiempo".getBytes();
    outDatagram = new DatagramPacket(outBuffer,outBuffer.length,
     localAddress,2345);
    socket.send(outDatagram);
    System.out.println("\nEnvia peticion de tiempo a "+localHost+" al puerto 2345.");
    socket.receive(inDatagram);
    InetAddress destAddress = inDatagram.getAddress();
    String destHost = destAddress.getHostName().trim();
    int destPort = inDatagram.getPort();
    System.out.println("Recibiendo un datagrama desde "+destHost+" at port "+
     destPort+".");
    String data = new String(inDatagram.getData());
    data=data.trim();
    System.out.println("El datagrama contiene los sig. datos: "+data);
   }
   outBuffer = new byte[bufferLength];
   outBuffer = "salida".getBytes();
   outDatagram = new DatagramPacket(outBuffer,outBuffer.length,
    localAddress,2345);
   socket.send(outDatagram);
  }catch (IOException ex){
   System.out.println("excepci¢n: IOException");
  }
 }
}
\end{minted}
\caption{Ejemplo de uso de datagramas. \textit{GetTimeApp}.}
% \caption[Long Code Example]{A long code example which will break across pages.}
\label{listing:1}
\end{longlisting}

% TABLA con más de una línea 
\begin{center}
\begin{tabular}{|m{\textwidth}| }
 \hline 
\rowcolor{gray!50} 
\begin{center}
\textbf{Ejercicios sugeridos} 
\end{center} 
\\ \hline

\begin{itemize}
\item      • ¿Qué pasa si al programa \textit{ReverServerApp} se tratan de conectar dos o más clientes? Proponga e implemente una solución para recibir más de un cliente. 
\item     • Modifique el ejemplo de uso de datagramas para poder conectarse desde cualquier máquina en la red al servidor \textit{TimeServerApp}.
\end{itemize}

\\ \hline
\end{tabular}
\end{center}


%%%%%%%%%%
% JAVA_END
%%%%%%%%%%



%%%%%%%%%%
% JAVA
%%%%%%%%%%
