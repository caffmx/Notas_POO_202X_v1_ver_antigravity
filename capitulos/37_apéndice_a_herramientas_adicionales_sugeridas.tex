\chapter*{Apéndice A. Herramientas adicionales sugeridas }
\addcontentsline{toc}{chapter}{Apéndice A}

%%%%%%%%%%
% JAVA
%%%%%%%%%%
\ifjava
\section*{BlueJ} \label{bluej}

BlueJ \footnote{Ver: \url{http://www.bluej.org/}} es un programa desarrollado por la universidades de \textit{Kent} y \textit{Deakin} para ayudar a los estudiantes a entender programación orientada a objetos en Java, particularmente ayuda a entender la herencia.

A partir de un diagrama de clases, BlueJ puede generar el código básico de la clase en Java, el cuál puede ser editado y compilado conforme las necesidades del programa. El programa es básico -ver figura \ref{bl} - y fácil de usar permitiendo entender estructuras complejas en las relaciones de herencia.

\begin{figure}
    \centering
    \includegraphics[scale=.7]{imagenes/BlueJ.png}
    \caption{ Interface de BlueJ }
    \label{fig:bluej}
\end{figure}	

El código Java generado por BlueJ para el diagrama de la figura anterior es el siguiente:

%longlist  para potencialmente grandes codigos fuentes
\begin{longlisting}
\begin{minted}
    [frame=lines,
framesep=2mm,
baselinestretch=1.2,
bgcolor=lightgray,
fontsize=\footnotesize]
    {java}
/**
 * Write a description of class Vehiculo here.
 * 
 * @author (your name) 
 * @version (a version number or a date)
 */
public class Vehiculo
{
	// instance variables - replace the example below with your own
	private int x;

	/**
	 * Constructor for objects of class Vehiculo
	 */
	public Vehiculo()
	{
		// initialise instance variables
		x = 0;
	}

	/**
	 * An example of a method - replace this comment with your own
	 * 
	 * @param  y   a sample parameter for a method
	 * @return     the sum of x and y 
	 */
	public int sampleMethod(int y)
	{
		// put your code here
		return x + y;
	}
}



/**
 * Write a description of class Camion here.
 * 
 * @author (your name) 
 * @version (a version number or a date)
 */
public class Camion extends Vehiculo
{
	// instance variables - replace the example below with your own
	private int x;

	/**
	 * Constructor for objects of class Camion
	 */
	public Camion()
	{
		// initialise instance variables
		x = 0;
	}

	/**
	 * An example of a method - replace this comment with your own
	 * 
	 * @param  y   a sample parameter for a method
	 * @return     the sum of x and y 
	 */
	public int sampleMethod(int y)
	{
		// put your code here
		return x + y;
	}
}
\end{minted}
\caption{Ejemplo de código generado por BlueJ.}
% \caption[Long Code Example]{A long code example which will break across pages.}
\label{listing:1}
\end{longlisting}

%%%%%%%%%%
% JAVA_END
%%%%%%%%%%
\fi

%%%%%%%%%%
% CPP
%%%%%%%%%%
\ifcpp
\newpage
\section*{UMLGEC++}

El proyecto de desarrollo de esta herramienta CASE (UMLGEC ++)\cite{rumbaugh1991object, matias2003herramienta} soporta la notación UML\footnote{Información básica sobre UML puede ser vista en \cite{fernandez2002modelado}} para diagramas de clase y generación de código en C++, con una interfaz lo más completa y sencilla posible, ver figura \ref{fig:umlgec}. Siendo útil para entender gráficamente conceptos básicos de objetos y su correspondiente implementación en código. Los elementos de este software son:

\begin{itemize}
\item Depósito de datos
\item Módulo para Creación de Diagramas y Modelado 
\item Generador de código
\item Analizador de sintaxis
\end{itemize}

\begin{figure}
    \centering
    \includegraphics[scale=.5]{imagenes/UMLGEC.png}
    \caption{ Herramienta CASE UMLGEC++ }
    \label{fig:umlgec}
\end{figure}

De la generación de código se puede decir:


\begin{itemize}
\item A partir del diagrama se genera la estructura de las clases.
\item Se crean automáticamente: el constructor, el constructor de copia, el operador de asignación, las operaciones de igualdad y el destructor. 
\item Todos los atributos y asociaciones son establecidos como privados independientemente de la visibilidad establecida por el usuario, pero el acceso a ellos está permitido mediante operaciones \textit{get} y \textit{set} generadas automáticamente para cada atributo o asociación, las cuáles adquieren la visibilidad correspondiente al atributo  o asociación al que hacen referencia.
\item Se definen los cuerpos de las operaciones \textit{get} y \textit{set}, como funciones \textit{inline}.
\end{itemize}
    
%%%%%%%%%%
% CPP_END
%%%%%%%%%%
\fi





%%%%%%%%%%
% DRAFT
%%%%%%%%%%
\newpage


